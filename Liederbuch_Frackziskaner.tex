\documentclass[twoside,8pt]{scrartcl}
\usepackage{scrhack} % Dieses Paket ändert Anweisungen und Definitionen anderer Pakete (zum Beispiel von Paketen dessen Entwicklung eingestellt wurden), damit sie besser mit dem KOMA-Script zusammenarbeiten
\usepackage[ngerman]{babel}
\usepackage[a6paper]{geometry}
\usepackage{lipsum}
\usepackage{graphicx}
\usepackage{fontenc}
\usepackage{lmodern}
\usepackage{xcolor} 
\usepackage{alnumsec}
\usepackage{chngcntr}

\surroundarabic[(][)]{}{.}
\otherseparators{3}
\alnumsecstyle{aaaaa}

\renewcommand*{\titlepagestyle}{empty}

% \pagestyle{headings}

\geometry{a6paper,inner=10mm,outer=10mm, top=1cm, bottom=2cm} 

\renewcommand{\baselinestretch}{0.9}

%Farben
\definecolor{color2}{rgb}{0.45,0.45,0.45}    % Dunkles Grau (zum Abheben) 

% Titelseite
\title{
	\fontfamily{ptm}{
		\fontsize{38}{40}
		\selectfont                % Schriftgröße für Namen 
		\color{color2}
		Liederbuch
	}
	\\ \includegraphics[scale=1.2]{bilder/Frackziskaner_Logo_einzeln.pdf}
}
\author{
	\fontfamily{bch}{
		\fontsize{20}{40}
		\selectfont                % Schriftgröße für Namen 
		\color{color2}
		%Frackziskaner\\ Satzung
		Frackziskaner  
		%	\vspace{-2cm}
	}
}
\date{\today\\
	\fontfamily{bch}{
		\fontsize{8}{40}
		\selectfont                % Schriftgröße für Namen 
Auflage 1}}
	
\begin{document}
\maketitle
\cleardoublepage
\begingroup
\parskip=0pt
\tableofcontents
\endgroup
\clearpage

\section{El-Bimbo}
El Bimbo ist ein Mann,\\
Der tausend tolle Dinge kann.\\
In San Juan, da hat er's leicht,\\
Denn wie er küsst, ist unerreicht.\\
\newline
El Bimbo sah mich an,\\
Und so fing die Geschichte an.\\
Doch ich hab' gleich gesagt: "Mach Schluss,\\
Denn nun gehört mir jeder Kuss!\\
Und wenn du küssen willst, dann komm zu mir;\\
Die schönsten Küsse schenk' ich dir,\\
Nur dir, nur dir allein, allein, nur dir, nur dir allein!"\\
\newline
El Bimbo lief mir weg.\\
Erst hatt' ich einen großen Schreck.\\
In San Juan, da ist kein Mann,\\
Der so wie Bimbo laufen kann.\\
Doch ich hab' nur gelacht und mir gedacht:\\
Ganz ohne mich hält er's nicht aus,\\
Er kommt schon bald zurück, zurück zu mir, zu mir nach Haus'.\\
\newline
Lalalalala...\\
\newline
El Bimbo kam zurück\\
Und sah mich an mit frohem Blick.\\
Er sprach: "Ich habe eingeseh'n,\\
Es ist bei dir allein nur schön."\\
Und hier in San Juan, da ist kein Mann,\\
Der so wie Bimbo lieb sein kann.\\
Denn so wie ihm erging's schon manchem großen Don Juan.\\
\newline
Lalala lalala...

\clearpage
\section{Cantina Band}
\textbf{Refrain}\\
Dankeschön, wir sind die Cantina Band. Wenn ihr Songwünsche habt, ruft sie einfach!\\\\
Spielt den selben Song nochmal! \\\\
Alles klar! Den selben Song und los!\\\\
\textbf{Refrain}...\\\\
\textbf{Refrain}...\\\\
\textbf{Refrain}...\\\\
\textbf{Refrain}...\\\\
\textbf{Refrain}...\\\\
\textbf{Refrain}...\\\\
\textbf{Refrain}...\\\\
\textbf{Refrain}...\\\\
\textit{letzter Refrain nach 10 Stunden\\\\
	Bei einer Länge von 35 Sekunden wird der Refrain 1029 mal wiederholt!}
\clearpage
\section{Wildeshauser Lied}
\textbf{Strophe 1}\\
Im schönen Oldenburger Lande,\\
da liegt ein Städtchen friedlich still,\\
Zwar nennt man stets es nur das alte,\\
doch weilt ein jeder gern darin,\\
Denn seine Bürger, bieder, brav,\\
Bring' stets ein offenes Herz dir dar,\\
Drum grüß' mit mir mein Wildeshausen,\\
Ein Hoch bring' meinem Heimatort.\\
\newline\\
\textbf{Strophe 2}\\
Gar friedlich liegt es in dem Tale,\\
Umkränzt von dunklem Fichtenhain,\\
Die alte Hunte plätschert leise,\\
entlang dem grünen Wiesenrain.\\
Die Saatenfelder rings umher,\\
Sie neigen grüßend ihre Ähr'n\\
Und geben jedem Wand're Kunde,\\
Von Wildeshauser Bürgerfleiß.\\
\newline\\
\textbf{Strophe 3}\\
Gemütlich sind auch die Bewohner,\\
Sie lieben einen frischen Trunk,\\
Und nach Tages Last und Mühen\\
vereint sie eine frohe Rund.\\
Bei Kologen Willi in dem Saal,\\
Bei Panschar und in Benecke's Hall\\
Und auch in Stegemann's Lokale,\\
Erschallt manch' Lied und freies Wort.\\\clearpage
\textbf{Strophe 4}\\
Und kommt dann erst zur Zeit der Pfingsten\\
Das alte liebe Schützenfest,\\
Greift alles nach den alten Flinten,\\
Großvaters Hut wird aufgesetzt.\\
General und Oberst steigen zu Pferd,\\
Die Mannschaft präsentiert's Gewehr,\\
Und nach bewährtem Trommelklang,\\
Marschiert man hin zur Vogelstang'\\
\newline
\textbf{Strophe 5}\\
Ja, alt ist unser Wildeshausen,\\
Ja, alt ist unser Schützenfest,\\
Drum lassen wir es uns nicht rauben,\\
Wir halten treu am Alten fest !\\
Und nach der Väter altem Brauch\\
Kredenzen wir den "Willkomm" auch.\\
Drum: Hoch das alte Wildeshausen !\\
Ein Hoch dem alten Schützenfest !	\\

\clearpage
\section{Heil Dir O Oldenburg}
\textbf{Strophe 1}\\
Heil dir O Oldenburg,\\
Heil deinen Farben,\\
Gott schütz dein edles Roß,\\
er segne deine Garben. \\
Wie Deine Eichen stark,\\
wie frei des Meeres Flut,\\
sei deutscher Männerkraft dein höchstes Gut.\\
\newline
\textbf{Strophe 2}\\
Ehr deine Blümelein,\\
pfleg ihre Triebe,\\
blau und rot blühen sie,\\
die Freundschaft und die Liebe. \\
Schleudert den fremden Kiel\\
der Sturm an deinen Strand\\
birgt ihn der Lotsen Schar mit treuer Hand.\\
\newline
\textbf{Strophe 3}\\
Wer deinem Herde naht,\\
fühlt augenblicklich,\\
daß er hier heimisch ist.\\
Er preiset sich so glücklich.\\
Führt ihn sein Wanderstab auch alle Länder durch,\\
du bleibst sein liebstes Land. Mein Oldenburg.\\


\clearpage
\section{Niedersachsenlied}
\textbf{Strophe 1}\\
Von der Weser bis zur Elbe, von dem Harz bis an das Meer\\
stehen Niedersachsens Söhne: eine feste Burg und Wehr.\\
Fest wie unsere Eichen halten alle Zeit wir Stand,\\
wenn Stürme brausen übers deutsche Vaterland.\\
Wir sind die Niedersachsen, sturmfest und erdverwachsen,\\
Heil Herzog Wittekinds Stamm.\\
\newline
\textbf{Strophe 2}\\
Wo fielen die Römischen Schergen? Wo versank die welsche Brut?\\
In Niedersachsens Bergen, an Niedersachsens Wut.\\
Wer warf den römschen Adler nieder in den Sand?\\
Wer hielt die Freiheit hoch im deutschen Vaterland?\\
Das warn die Niedersachsen, sturmfest und erdverwachsen,\\
Heil Herzog Wittekinds Stamm.\\
\newline
\textbf{Strophe 3}\\
Auf blühend roter Heide starben einst viel tausend Mann,\\
für Niedersachsens Treue traf sie des Franken Bann.\\
Viel tausend Brüder fielen von des Henkers Hand.\\
Viel tausend Brüder für ihr Niedersachsenland.\\
Das warn die Niedersachsen, sturmfest und erdverwachsen,\\
Heil Herzog Wittekinds Stamm.\\
\newline
\textbf{Strophe 4}\\
Aus der Väter Blut und Wunden wächst der Söhne Heldenmut.\\
Niedersachsen solls bekunden: Für die Freiheit Glut und Blut!\\
Fest wie unsere Eichen halten alle Zeit wir stand,\\
wenn Stürme brausen übers deutsche Vaterland.\\
Wir sind die Niedersachsen, sturmfest und erdverwachsen.\\
Heil Herzog Wittekinds Stamm.\\

\clearpage
\section{Nationalhymne}
\textbf{Strophe 3}\\
Einigkeit und Recht und Freiheit\\
für das deutsche Vaterland!\\
Danach lasst uns alle streben\\
brüderlich mit Herz und Hand!\\
Einigkeit und Recht und Freiheit\\
sind des Glückes Unterpfand.\\
Blüh im Glanze dieses Glückes,\\
Blühe, deutsches Vaterland!

\clearpage
\section{Potpourrie}
\textbf{Wildeshauser Lied - Strophe 1}\\
Im schönen Oldenburger Lande,\\
da liegt ein Städtchen friedlich still,\\
Zwar nennt man stets es nur das alte,\\
doch weilt ein jeder gern darin,\\
Denn seine Bürger, bieder, brav,\\
Bring' stets ein offenes Herz dir dar,\\
Drum grüß' mit mir mein Wildeshausen,\\
Ein Hoch bring' meinem Heimatort.\\\\
\textbf{Heil dir O Oldenburg - Strophe 1}\\
Heil dir O Oldenburg,\\
Heil deinen Farben,\\
Gott schütz dein edles Roß,\\
er segne deine Garben. \\
Wie Deine Eichen stark,\\
wie frei des Meeres Flut,\\
sei deutscher Männerkraft dein höchstes Gut.\\\\
\textbf{Niedersachsenlied - Strophe 1}\\
Von der Weser bis zur Elbe, von dem Harz bis an das Meer\\
stehen Niedersachsens Söhne: eine feste Burg und Wehr.\\
Fest wie unsere Eichen halten alle Zeit wir Stand,\\
wenn Stürme brausen übers deutsche Vaterland.\\
Wir sind die Niedersachsen, sturmfest und erdverwachsen,\\
Heil Herzog Wittekinds Stamm.\\

\clearpage
\section{Wenn ich einmal traurig bin}
\textbf{Strophe 1}
\newline
Immer wenn ich traurig bin trink ich einen Korn\\
Wenn ich dann noch traurig bin trink ich noch n Korn\\
Wenn ich dann noch traurig bin trink ich noch n Korn\\
Und wenn ich dann noch traurig bin fang ich an von vorn.\\

\clearpage
\section{Die kleine Kneipe}
\textbf{Strophe 1}
\newline
Der Abend senkt sich auf die Dächer der Vorstadt.\\ 
Die Kinder am Hof müssen heim.\\ 
Die Krämersfrau fegt das Trottoir vor dem Laden,\\ 
ihr Mann trägt die Obstkisten rein.\\
Der Tag ist vorüber, die Menschen sind müde.\\
Doch viele gehen nicht gleich nach Haus,\\
denn drüben klingt aus einer offenen\\
Türe Musik auf den Gehsteig hinaus.\\
Die kleine Kneipe in unserer Straße, \\
da, wo das Leben noch lebeswert ist,\\
dort, in der Kneipe in unserer Straße. \\
Da fragt Dich keiner, was Du hast oder bist.\\
\newline
\textbf{Strophe 2}
\newline
Die Postkarten dort an der Wand in der Ecke,\\
das Photo vom Fußballverein, das Stimmengewirr,\\
die Musik aus der Jukebox all das ist ein Stückchen daheim\\
du wirfst eine Mark in den Münzautomat,\\
Schaust andern beim Kartenspiel zu,\\
 und stehst mit dem Pils in der Hand an der Theke,\\
und bist gleich mit jedem per Du. \\
Die kleine Kneipe in unserer Straße,\\
da, wo das Leben noch lebeswert ist, \\
dort, in der Kneipe in unserer Straße.\\
Da fragt Dich keiner, was Du hast oder bist.\\
\clearpage
\textbf{Strophe 3}
\newline
Man redet sich heiß und spricht sich von der Seele,\\
was einem die Laune vergällt.\\
Bei Korn und bei Bier findet\\
mancher die Lösung für alle Probleme der Welt\\
Wer Hunger hat, der bestellt Würstchen mit Kraut,\\
Weil es andere Speisen nicht gibt.\\
Die Rechnung, die steht auf dem Bierdeckel drauf.\\
Doch beim Wirt hier hat jeder Kredit.\\
Die kleine Kneipe in unserer Straße,\\
da, wo das Leben noch lebeswert ist,\\
dort, in der Kneipe in unserer Straße. \\
Da fragt Dich keiner, was Du hast oder bist.\\
Die kleine Kneipe in unserer Straße,\\
 da, wo das Leben noch lebeswert ist,\\
dort, in der Kneipe in unserer Straße. \\
Da fragt Dich keiner, was Du hast oder bist.\\



\clearpage
\section{Der Teufel hat den Schnaps gemacht}

\textbf{Strophe 1}\newline
Als ich in meiner Kneipe saß\\
kam ein Mädchen durch die Tür\\
sie sah erst mich und dann mein Glas\\
und setzte sich zu mir.\\
Sie sagte\\
sie sei von der Heilsarmee\\
und kann mich nicht verstehn\\
denn sie sah schon manch' braven Mann\\
hier vor die Hunde gehn.\\
\newline
\textbf{Refrain}\newline
Der Teufel hat den Schnaps gemacht\\
um uns zu verderben.\\
Ich hör' schon\\
wie der Teufel lacht\\
wenn wir am Schnaps einmal sterben.\\
%
\newline
\textbf{Strophe 2}
\newline
Sie war so fromm\\
sie war so lieb\\
und sie gefiel mir gut\\
und freundlich hab' ich ihr erklärt\\
daß mir der Schnaps nichts tut.\\
Schon leerte ich das nächste Glas.\\
Sie sprach: "Du tust mir leid\\
denn mancher\\
der so säuft wie du\\
hat's später dann bereut".\\
\newline
\textbf{Refrain}...\\
%
\clearpage
\textbf{Strophe 3}
\newline
Das Mädchen sah mich zärtlich an\\
drum trank ich schnell noch aus\\
ich legte meinen Arm um sie\\
und brachte sie nach Haus.\\
Sie lud mich in ihr Zimmer ein\\
und dort erfuhr ich dann:\\
wer zuviel trinkt\\
ist leider oft\\
nur noch ein halber Mann.\\
\newline
\textbf{Refrain}...




\clearpage
\section{Nehmt Abschied Brüder}
\textbf{Strophe 1}\\
Nehmt Abschied Brüder,\\
ungewiss ist alle wiederkehr\\
Die Zukunft liegt in Finsternis\\
%
Und macht das Herz uns schwer\\
\newline
\textbf{Refrain}\\
Der Himmel wölbt sich übers Land\\
Ade, Auswiedersehn\\
Wir ruhen all in Gottes Hand\\
Lebt wohl, Aufwiedersehn\\
\newline
\textbf{Strophe 2}\\
Die Sonne sinkt, es steigt die Nacht\\
Vergangen ist der Tag\\
Die Welt schläft ein und leis erwacht\\
Der Nachtigallenschlag\\
\newline
\textbf{Refrain}\\
\newline
\textbf{Strophe 3}\\
So ist in jedem Anbeginn\\
Das Ende nicht mehr weit\\
Wir kommen her und gehen hin\\
Und mit uns geht die Zeit\\
\newline
\textbf{Strophe 4}\\
Nehmt Abschied Brüder\\
Schließt den Kreis\\
Das Leben ist ein Spiel\\
Und wer es recht zu Spielen weiss\\
Gelangt ans große Ziel\\
Der Himmel wölbt sich übers Land\\
Ade, Au\\


\section{Niedersachsenlied (Frieda \& Anneliese) feat. Frackziskaner}
\textbf{Strophe 1}
\newline
Unterm Acker liegt der Opa,\\
oben drauf da wächst der Mais,\\
und das Geld kommt aus Europa,\\
der Politik sei Lob und Preis.\\
\newline
Wo sind die Schweinehälften\\
halb so groß wien Schwein?\\
Wo sind die Hühnerstelle\\
für die Hühner viel zu klein?\\
Hier bei uns in Niedersachsen\\
schmiert man die Treckerachse\\
noch selber ab mit der Hand.\\
\newline
\textbf{Strophe 2}
\newline
Wo trinkt man Lüttje Lage,\\
kommt in die Cola noch der Korn?\\
Wo hat so mancher Magen,\\
seinen Inhalt schnell verloren?\\
\newline
Wo sind die Schweinehälften\\
halb so groß wien Schwein?\\
Wo sind die Hühnerstelle\\
für die Hühner viel zu klein?\\
Hier bei uns in Niedersachsen,\\
wo Hanf und Honig wachsen,\\
da melken wir noch mit der Hand.\\



\clearpage
\section{Tippelbrüder-Polka}

Ob der Himmel blau, ob die Wolken sind so grau,\\
zieh'n wir durch die Welt, tippeln wie's uns gefällt\\
Wenn auch die Zeit so schnell vergeht,\\
niemals kommen wir zu spät. \\
Brüder laßt uns wandern, die Welt ist so schön.\\
\newline
Spielleute sind wir, musizieren dort und hier,\\
froh und mit viel Schwung, denn das erhält uns jung.\\
Ob flotter Marsch, ob Potpourri,\\
Langeweile gibt es nie,\\
lust'gen Musikanten gehöret die Welt\\
\newline
Hört mal alle zu, wer wir sind und was wir tun,\\
wer wir sind ist klar, Spielleut aus Lippetal.\\
Und was wir tun, das könnt ihr seh'n,\\
feiern, so lange wir noch steh'n.\\
Brüder laßt uns feiern, die Welt ist so schön.\\

\clearpage
\section{Oh du schöner Westerwald}

\textbf{Strophe 1}\newline
Heute wollen wir marschieren\\
einen neuen Marsch probieren\\
in dem schönen Westerwald\\
ja da pfeift der Wind so kalt.\\
\newline
\textbf{Refrain}\newline
Oh du schöner Westerwald\\
Über deine Höhen pfeift der Wind so kalt\\
jedoch der kleinste Sonnenschein\\
dringt tief ins Herz hinein\\
\newline
\textbf{Strophe 2}\newline
Und die Gretel und der Hans\\
gehn des Sonntags gern zum Tanz\\
weil das Tanzen Freude macht\\
und das Herz im Leibe lacht\\
\newline
\textbf{Refrain}...\\
\newline
\textbf{Strophe 3}\newline
Ist das Tanzen dann vorbei\\
gibts gewöhnlich Keilerei\\
und vom Bursch den das nicht freut\\
sagt man “Der hat kein Schneid.”\\
\newline
\textbf{Refrain}...

\clearpage
\section{Es gibt kein Bier auf Hawaii}
\textbf{Refrain}\\
Es gibt kein Bier auf Hawaii\\
es gibt kein Bier\\
drum fahr ich nicht nach Hawaii\\
drum bleib ich hier\\
es ist so heiß auf Hawaii\\
kein kühler Fleck\\
und nur vom Hulahula geht der Durst nicht weg\\
\newline
\textbf{Strophe 1}\newline
Meine Braut sie heißt Marianne\\
wir sind seit 12 Jahren verlobt\\
sie hätt mich so gern zum Manne\\
und hat schon mit Klage gedroht\\
die Hochzeit wär längst schon gewesen\\
wenn die Hochzeitsreise nicht wär\\
denn sie will nach Hawaii, ja sie will nach Hawaii\\
und das fällt mir so unsagbar schwer\\
\newline
\textbf{Refrain}...\\
\newline
\textbf{Strophe 2}\newline
wenn sie mit nach Pilsen führe\\
ja, dann wären wir längst schon ein Paar\\
doch all meine Bitten und Schwören\\
verschmähte sie Jahr um Jahr\\
sie singt Tag und Nacht neue Lieder\\
von den Palmen am blauen Meer\\
denn sie will nach Hawaii, ja sie will nach Hawaii\\
und das fällt mir so unsagbar schwer\\
\newline
\textbf{Refrain}...
\clearpage
\section{Bommerlunder eisgekühlt}
Bommerlunder eisgekühlt\\
Eisgekühlter Bommerlunder -\\
Bommerlunder eisgekühlt\\
\newline
Und dazu:\\
Ein belegtes Brot mit Schinken - Schinken!\\
Ein belegtes Brot mit Ei - Ei!\\
Das sind zwei belegte Brote\\
Eins mit Schinken uns eins mit Ei\\
\newline
Und dazu:\\
Eisgekühlter Bommerlunder -\\
Bommerlunder eisgekühlt\\
Eisgekühlter Bommerlunder -\\
Bommerlunder eisgekühlt\\
\newline
Und dazu:\\
Ein belegtes Brot mit Schinken - Schinken!\\
Ein belegtes Brot mit Ei - Ei!\\
Das sind zwei belegte Brote\\
Eins mit Schinken uns eins mit Ei\\
\clearpage
\section{I'm Gonna Be (500 Miles)}
\textbf{Strophe 1}\newline
When I wake up, well I know I'm gonna be\\
I'm gonna be the man who wakes up next to you\\
When I go out, yeah I know I'm gonna be\\
I'm gonna be the man who goes along with you\\
If I get drunk, well I know I'm gonna be\\
I'm gonna be the man who gets drunk next to you\\
And if I haver, hey I know I'm gonna be\\
I'm gonna be the man who's havering to you\\
\newline
\textbf{Refrain}\\
But I would walk 500 miles\\
And I would walk 500 more\\
Just to be the man who walks a thousand miles\\
To fall down at your door\\
\newline
\textbf{Strophe 2}\newline
When I'm working, yes I know I'm gonna be\\
I'm gonna be the man who's working hard for you\\
And when the money comes in for the work I do\\
I'll pass almost every penny on to you\\
When I come home, oh I know I'm gonna be\\
(When I come home)\\
I'm gonna be the man who comes back home to you\\
And if I grow old, well I know I'm gonna be\\
I'm gonna be the man who's growing old with you\\
\newline
\textbf{Refrain}...\\
\newline
Da d-da da, da d-da da, da d-da da, da d-da da\\
Da-da-da dun-diddle un-diddle un-diddle a da da\\
Da d-da da, da d-da da, da d-da da, da d-da da\\
Da-da-da dun-diddle un-diddle un-diddle a da da\\
\clearpage
\textbf{Strophe 3}\newline
When I'm lonely, well I know I'm gonna be\\
I'm gonna be the man who's lonely without you\\
And when I'm dreaming, well I know I'm gonna dream\\
I'm gonna dream about the time when I'm with you\\
When I go out, well I know I'm gonna be\\
(When I go out)\\
I'm gonna be the man who goes along with you\\
And when I come home, yes I know I'm gonna be\\
(When I come home)\\
I'm gonna be the man who comes back home with you\\
I'm gonna be the man who's coming home with you\\
\newline
\textbf{Refrain}...\\
\newline
Da d-da da, da d-da da, da d-da da, da d-da da\\
Da-da-da dun-diddle un-diddle un-diddle a da da\\
Da d-da da, da d-da da, da d-da da, da d-da da\\
Da-da-da dun-diddle un-diddle un-diddle a da da\\
\newline
Da d-da da, da d-da da, da d-da da, da d-da da\\
Da-da-da dun-diddle un-diddle un-diddle a da da\\
Da d-da da, da d-da da, da d-da da, da d-da da\\
Da-da-da dun-diddle un-diddle un-diddle a da da\\
\newline
\textbf{Refrain}...\\
\clearpage
\section{Sto Lat}
Sto lat! Sto lat!\\
Niech zyje, zyje nam. \\
Sto lat! Sto lat!\\
Niech zyje, zyje nam. \\
Jeszcze raz! Jeszcze raz! Niech zyje, zyje nam. \\
Niech zyje nam. \\
\newline
Niech ci? gwiazdka pomyslnosci nigdy nie zagasnie, \\
nigdy nie zagasnie!\\ 
A kto z nami nie wypije,\\
niech go piorun trzasnie. \\
A kto z nami nie wypije,\\
niech go piorun trzasnie. \\
\newline
Sto lat! Sto lat! Sto lat!\\
Sto lat! Niechaj zyje nam. \\
Sto lat! Sto lat! Sto lat!\\
Sto lat! Niechaj zyje nam. \\
Niech zyje nam! Niech zyje nam! \\
Zdrowia, szczescia, pomyslnosci! \\
Niechaj zyj\\
\clearpage
\section{Werder Bremen - Wo die Weser einen großen Bogen}
Wo die Weser einen großen Bogen macht, \\
wo das Weser-Stadion strahlt in neuer Pracht, \\
wo man trägt die allerschönsten Spiele aus \\
da ist Werder Bremen, da sind wir zu Haus; \\
da ist Werder Bremen, da sind wir zu Haus. \\
\newline
Wir steh'n für Werder ein, für SV Werder ein, \\
für Werder Bremen, unser'n Verein. \\
Wir steh'n für Werder ein, das soll der Schlachtruf sein, \\
für Werder Bremen, unser'n Verein. \\
\newline
Wo die Weser einen großen Bogen macht, \\
rollt, ja rollt das Leder, dass es nur so kracht. \\
\newline
\textit{Und das ganze Weser-Stadion singt im Chor:} \\
\newline
Fußball unser Leben, Werder noch ein Tor; \\
Fußball unser Leben, Werder noch ein Tor. \\
\newline
Wir steh'n für Werder ein, für SV Werder ein, \\
für Werder Bremen, unser'n Verein. \\
Wir steh'n für Werder ein, das soll der Schlachtruf sein, \\
für Werder Bremen, unser'n Verein. \\
\newline
Wir steh'n für Werder ein, für SV Werder ein, \\
für Werder Bremen, unser'n Verein. \\
Wir steh'n für Werder ein, das soll der Schlachtruf sein, \\
für Werder Bremen, unser'n Verein.\\
\clearpage
\section{Auf der Reeperbahn nachts um halb eins}
\textbf{Strophe 1}\newline
Silbern klingt und springt die Heuer,\\
heut' speel ick dat feine Oos.\\
Heute ist mir nichts zu teuer,\\
morgen geht die Reise los.\\
Langsam bummel ich ganz alleine\\
die Reeperbahn nach der Freiheit 'rauf,\\
treff ich eine recht blonde, recht feine,\\
die gabel ich mir auf.\\
\newline
\textbf{Refrain}\newline
Komm doch, liebe Kleine, sei die meine, sag' nicht nein!\\
Du sollst bist morgen früh um neune meine Herz allerliebste sein.\\
Ist dir's recht, na dann bleib' ich dir treu sogar bis um zehn.\\
Hak' mich unter, wir wollen jetzt zusammen mal bummeln geh'n.\\
\newline
\textbf{Strophe 2}\newline
Auf der Reeperbahn nachts um halb eins,\\
ob du'n Mädel hast oder hast kein's,\\
amüsierst du dich,\\
denn das findet sich\\
auf der Reeperbahn nachts um halb eins.\\
Wer noch niemals in lauschiger Nacht\\
einen Reeperbahnbummel gemacht,\\
ist ein armer Wicht,\\
denn er kennt dich nicht,\\
mein Sankt Pauli, Sankt Pauli bei Nacht.\\
\newline
Kehr ich heim im nächsten Jahre,\\
braungebrannt wie zo'n Hottentott;\\
hast du deine blonden Haare\\
schwarz gefärbt, vielleicht auch rot,\\
grüßt dich dann mal ein fremder Jung',\\
und du gehst vorüber und kennst ihn nicht,\\
kommt dir vielleicht die Erinnerung wieder,\\
wenn leis' er zu dir spricht:\\
\newline
\textbf{Refrain}...\\
\newline
Auf der Reeperbahn nachts um halb eins,\\
ob du'n Mädel hast oder hast kein's,\\
amüsierst du dich,\\
denn das findet sich\\
auf der Reeperbahn nachts um halb eins.\\
Wer noch niemals in lauschiger Nacht\\
einen Reeperbahnbummel gemacht,\\
ist ein armer Wicht,\\
denn er kennt dich nicht,\\
mein Sankt Pauli, Sankt Pauli bei Nacht.\\
\clearpage
\section{Every Sperm is sacred}
\textit{Gruppe:}\\
Every sperm is sacred.\\
Every sperm is great.\\
If a sperm is wasted,\\
God gets quite irate.\\
\newline
\textit{Jan Hartwig:}\\
Let the heathen spill theirs\\
On the dusty ground.\\
God shall make them pay for\\
Each sperm that can't be found.\\
\newline
\textit{Gruppe:}\\
Every sperm is wanted.\\
Every sperm is good.\\
Every sperm is needed\\
In your neighbourhood.\\
\newline
\textit{Jan Hartwig:}\\
Hindu, Taoist, Mormon,\\
Spill theirs just anywhere,\\
But God loves those who treat their\\
Semen with more care.\\
\clearpage
\section{Was wollen wir trinken}
Was wollen wir trinken,\\
sieben Tage lang,\\
was wollen wir trinken,\\
so ein Durst.\\
\newline
Es wird genug für alle sein,\\
wir trinken zusammen,\\
roll das Fass mal rein,\\
wir trinken zusammen,\\
nicht allein.\\
\newline
Dann wollen wir schaffen,\\
sieben Tage lang,\\
dann wollen wir schaffen,\\
komm fass an!\\
\newline
Und dass wird keine Plagerei,\\
wir schaffen zusammen,\\
sieben Tage lang,\\
ja schaffen zusammen,\\
nicht allein.\\
\newline
Jetzt müssen wir streiken,\\
keiner weiß wielang,\\
ja für ein Leben ohne Zwang\\
\newline
Dann kriegt der Frust uns nicht mehr klein,\\
wir halten zusammen,\\
keiner kämpft allein\\
wir gehen zusammen\\
nicht allein\\
\newline
lalalalalalalalala...
\clearpage
\section{Hej Sokoly}
Hej, hej, hej sokoly,\\
omijajcie, gory, lasy, doly.\\
Dzwon, dzwon, dzwon dzwoneczku\\
moj stepowy skowroneczku,\\
Hej, hej, hej sokoly,\\
omijajcie gory, lasy, doly.\\
Dzwon, dzwon, dzwon dzwoneczku,\\
moj stepowy dzwon, dzwon, dzwon.\\



\end{document}
